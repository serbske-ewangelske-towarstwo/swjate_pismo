\documentclass[twocolumn,b5paper]{book}

\begin{document}

\chapter*{Prěnje knihi Mójzaskowe}

\section*{1. staw}

\subsection*{\textit{Stworjenje swěta.}} 

1. W zpočatku ztwori Bóh njebjesa a zemju. \hfill {\footnotesize \textit{Jant, 1. Kók 1, 16. Pjz. 33, 6. Pz. 102, 26. Hebr. 11, 3.}}

2. A zemja běše puzta a prózna, a ćma bě na hłubokozći; a Boži Duch lětaše na wodach. 

3. A Bóh dźeše: Budź swětło! Dha bu swětło. \hfill {\footnotesize \textit{2 Kor. 4, 6.}}

4. A Bóh widźiše swětło, zo je dobre. Duž rosdźěli Bóh swětło wot ćmy. \hfill {\footnotesize \textit{Jez. 45, 7.}}

5. A Bóh mjenowaše to swětło dźeń, a tu ćmu mjenowaše wón nóz. A bu wječor a bu rano, prěni dźeń. 

6. A Bóh dźeše: Budź twjerdozć bjes wodami a dźěl wody wot wodow. \hfill {\footnotesize \textit{Jer. 10, 12. St. 51, 15. Pj. 186, 5.}}

7. A Bóh zčini tu twjerdozć a rosdźěli wody, kotrež su pod twjerdozću, wot wodow, kotrež su na twjerdozći. A zta so tak. \hfill {\footnotesize \textit{Pj. 104, 3. Pz. 148, 4.}}

8. A Bóh mjenowaše tu twjerdozć njebjesa. A bu wječor a bu rano, druhi dźeń. 

9. A Bóh dźeše: Shromadźće so wody pod njebjesami na jene mězto, zo by to suche widźene A zta so tak. \hfill {\footnotesize \textit{Hiob. 38, 8 Pz. 33, 7. Pj. 104, 7. o. Pz. 136, 6.}}

10. A Bóh mjenowaše to suche zemju, a to zhromadźenje wodow mjenowaše wón morjo. A Bóh widźiše, zo to běše dobre. 

11. A Bóh dźeše: Semja płodź trawu a zele, kotrež by dawało symjo, a płódne štomy, kiž kóždy by pšinjesł swojeho runja płód a měł swoje symjo pši bebi na zemi. A zta so tak. 

12. A zemja płodźeše trawu a zele, kotrež dawaše symjo, kóžde swojeho runja, a štomy, kotrež pšinjesu płód a swoje symjo pši sebi maju, kóždy swojeho runja. A Bóh widźiše, zo to běše dobre. 

13. A bu wječor a bu rano, tzeći dźeń. 

14. A Bóh dźeše: Budźće swězy na twjerdozći tych njebjesow, zo bychu rosdźěliłe dźeń wot nozy a dawałe znamjenja, časy, dny a lěta. \hfill {\footnotesize \textit{St. 8, 22. sj. 136, 7. Sir. 43, 2—9.}} 

15. A byłe swězy na twjerdozći tych njebjesow, zo bychu swěćiłe na zemju. A zta so tak. 

16. A Bóh ztwori dwě wulzy swězy: jenu wjetšu swězu, zo by wodnjo knježiła, a jenu mjeńšu swězu, zo by w nozy knježiła, a k temu tež hwězdy. \hfill {\footnotesize \textit{5 Mójz. 4, 19. Hiob. 9, 0.}} 

17. A Bóh je ztaji na twjerdozć tych njebjesow, zo bychu swěćiłe na zemju. 

18. A zo bychu knježiłe na dnju a na nozy a rosdźěliłe swětło a ćnu. A Bóh widźiše, zo to běše dobre. \hfill {\footnotesize \textit{Pz. 104, 20.}}

19. A bu wječor a bu rano, štwórty dźeń. 

20. A Bóh dźeše: Njech so wody mjeŕwja z płuwazymi a žiwymi zwěrjatami a z ptakami, kotrež na zemi pod twjerdozću tych njebjesow lětaju. \hfill {\footnotesize \textit{St. 2, 19.}}

21. A Bóh ztwori wulke móŕzke ryby a wšelake žiwe a łažaze zwěrjata, kiž so we wodźe mjeŕwja, kóžde swojeho runja, a wšelake lětaze ptaki, kóždy swojeho runja. A Bóh widźiše, zo to dobre běše. \hfill {\footnotesize \textit{Pz. 104, 26. Hiob. 40, 10,}} 

22. A Bóh požohnowaše je a dźeše: Płodźće so a pšizporjejće so a napjelńće tu wodu w morju, a ptaki, pšizporjejće so na zemi. \hfill {\footnotesize \textit{Scht. 28. St. 8, 17. St. 9, 1. 7.}}

23. A bu wječor a bu rano, pjaty dźeń. 

24. A Bóh dźeše: Semja pšinjes žiwe zwěrjata, kóžde swojeho runja; zkót, waki a zwěrjata na zemi, jene kóžde swojeho runja. A zta so tak. \hfill {\footnotesize \textit{Hiob. 12, 7. Sir. 16, 30.}}

25. ... 

\section*{2. staw}

\subsection*{\textit{Ssabat. Stworjenje člowjeka. Paradiszahroda. Božej pšikazni a mandźelztwo.}} 

1. Tak buchu dokonjane njebjesa a zemja a wšitko jich wójzko. 

2. A tak dokonja Bóh sedmy dźeń swoje dźěło, kotrež bě činił. A wotpočowaše sedmy dźeń wot wšitkeho swojeho dźěła, kotrež wón bě zčinił. \hfill {\footnotesize \textit{2 Mójz. 20, 11. St. 31, 17., 5, 14. Jez. 40, 28. Hebr. 4, 4.}} 

3. A Bóh požohnowaše tón sedmy dźeń a swjećeše jón, teho dla, zo wón běše na nim wotpočował wot wšitkeho swojeho dźěła, kotrež Bóh ztworił a zčinił bě. 

4. Tajke je to ztworjenje njebjesow a zemje, hdyž běchu ztworjene, w času, hdyž Bóh tón Knjes zemju a njebjesa ztwori. 

5. A wšelake štomy na polu, kotrež prjedy njejsu byłe na zemi, a wšelake zele na polu, kotrež prjedy nihdy njeje roztło. Pšetož Bóh tón Knjes hišće njeběše dał dešć hić na zemju, a njeje był člowjek, kiž by zemju dźěłał. 

6. Ale młha dźěše ze zemje a mačeše zyłu zemju. 

7. A Bóh tón Knjes ztwori člowjeka z procha teje zemje a dunu jemu žiwy dych do jeho nosa. A tak bu člowjek žiwa duša. \hfill {\footnotesize \textit{St. 1, 26. 1. Kor. 15, 45.}}

8. ...

\chapter*{Druhe knihi Mójzaskowe}

\section*{1. staw}

\subsection*{\textit{Jzraelzkich dźěći słužba a tyšnozć we Egiptowzkej.}} 

1. To su mjena Jzraelzkich dźěći, kotrež z Jakubom do Egiptowzkeje pšińdźechu; kóždy pšińdźe ze swojej khěžu nuts: \hfill {\footnotesize \textit{1 Mójz. 46, 8.}}

2. Ruben, Simeon, Levi, Juda, 

3. Jzašar, Sebulon, Benjamin, Mójzasowe. 

4. Dan, Naphthali, Gad, Aser. 

5. A wšitkich dušow, kotrež běchu z Jakubowych bjedrow pšišłe, běše sydom dźesać. Jozeph pak bě prjedy we Egiptowzkej. \hfill {\footnotesize \textit{1 Mójz. 46, 27.}}

6. Hdyž pak Jozeph wumrjeł bě a wšitzy jeho bratzja a wšitzy ći, kotziž su we tym času žiwi byli,

7. Rosrozćichu so Jzraelzke dźěći a płodźachu dźěći, pšizporjachu so a rossylnichu so jara wulzy, zo jich tón kraj połny bě. \hfill {\footnotesize \textit{Pj. 105, 24. Jap. zk. 7, 17.}}

8. ...

\end{document}

